%
% V1: 1 Feb. 2013
% V2: 22 Nov. 2020; Added notion for tensors. Moved to overleaf.com
% 
% C . Shen: 4:04:01 pm, 1st January, 2021 commented: 
%   
%
%
% remove space around paragraphs
%
\makeatletter
\renewcommand{\paragraph}{%
  \@startsection{paragraph}{4}%
  {\z@}{0.5ex \@plus 1ex \@minus .1ex}{-1em}%
  {\normalfont\normalsize\bfseries}%
}
\makeatother

%
% remove space around equations
%
\AtBeginDocument{%
 \abovedisplayskip=8pt plus 3pt minus 5pt
 \abovedisplayshortskip=0pt plus 3pt
 \belowdisplayskip=8pt plus 3pt minus 5pt
 \belowdisplayshortskip=5pt plus 3pt minus 4pt
}


%
% fonts; Do not use times package. Use the following
% 
\renewcommand{\rmdefault}{ptm}
\renewcommand{\sfdefault}{phv}

\usepackage[]{amsmath}                  % Handy tools for mathematical typesetting
\usepackage[]{mathtools}                % Mathtools package requires and fixes amsmath
\usepackage[]{amsthm}
\usepackage[]{amssymb,amsfonts}
\usepackage{graphicx}
\usepackage{bm}
\usepackage{bbm}

\usepackage[]{xcolor}
\usepackage[mathscr]{eucal}

\usepackage[]{xparse}
\usepackage[]{xstring}
\usepackage{etoolbox}
\usepackage[]{xspace}
\usepackage[]{comment}

\usepackage{nicefrac}

\usepackage[margin=4pt,font=small,labelfont=bf,labelsep=endash,tableposition=top]{caption}

\usepackage{multirow}

\usepackage{url}
\def\UrlFont{\rm\small\ttfamily}

 
%
% -- math symbols --
% http://tex.stackexchange.com/questions/83939/new-command-based-on-command-name
%
% Define vectors in boldface, lowercase; matrices in
% boldface, uppercase.
% Examples: This is a vector $\vx$ and this is a matrix
% $\mA$.  This is a vector $\vec{x}$ and this is a matrix
% $\mat{A}$.  Also $\mat{\Gamma}$ works.
%
% \DeclareMathAlphabet{\mathbmit}{OML}{cmm}{b}{it}               
% \DeclareMathAlphabet{<math-alph>}{<encoding>}{<family>}{<series>}{<shape>}                           %  it: italic;  n: `normal' math 
                  %
               
\renewcommand{\vec}[1]{\ensuremath{\pmb{#1}}}
\newcommand{\mat}[1]{\ensuremath{\mathbf{#1}}}
\newcommand{\set}[1]{\ensuremath{\mathscr{#1}}}

%\newcommand{\tensor}[1]{\ensuremath{\pmb{ \mathtt{ #1} }  }}

\newcommand{\tensor}[1]{\ensuremath{ {\sf #1}\hspace{-0.24659cm} {\sf #1} }}




\makeatletter
\@tfor\next:=abcdefghijklmnopqrstuvwxyxz\do
{\begingroup\edef\x{\endgroup
        \noexpand\@namedef{v\next}{\noexpand\vec{\next}}%
    }\x}
%
\@tfor\next:=ABCDEFGHIJKLMNOPQRSTUVWXYZ\do
{\begingroup\edef\x{\endgroup
        \noexpand\@namedef{m\next}{\noexpand\mat{\next}}%
    }\x}
%
\@tfor\next:=ABCDEFGHIJKLMNOPQRSTUVWXYZ\do
{\begingroup\edef\x{\endgroup
        \noexpand\@namedef{s\next}{\noexpand\set{\next}}%
    }\x}
%
% tensor
%
\@tfor\next:=ABCDEFGHIJKLMNOPQRSTUVWXYZ\do
{\begingroup\edef\x{\endgroup
        \noexpand\@namedef{t\next}{\noexpand\tensor{\next}}%
    }\x}
\makeatother


\def\sG{\mathcal{G}}                                % Override \sG to make it look nicer;

\newcommand{\Transpose}{\ensuremath{{\!\top}} }     % Transpose ^T
\let\T\Transpose


\newcommand{\opt}{\ensuremath{\textsc{Opt}}}

\newcommand{\norm}[1]{\ensuremath{\left\|#1 \right\|}}

\newcommand{\eyes}{\ensuremath{{\mathbf{I}}}}

% Set indicator function
\newcommand{\1}{{\mathbbm{1}}}           
                                                



% getting a dot the size between \cdot and \bullet
% \newcommand\bcdot{\ensuremath{\vcenter{\hbox{\tiny$\bullet$}}}}
\newcommand{\bcdot}{\raisebox{1pt}{\textbf{\large .}}}
%

% smaller than \bcdot
\newcommand{\sdot}{\raisebox{1pt}{{.}}}



\noexpandarg
\newcommand\col[1]{%
\StrBefore{#1}{_}[\matrix]%
\StrBehind{#1}{_}[\colindex]%
{\ensuremath{{\matrix}_{ \bcdot    \mkern-1.5mu    \colindex}}}%
}

\noexpandarg
\newcommand\row[1]{%
\StrBefore{#1}{_}[\matrix]%
\StrBehind{#1}{_}[\rowindex]%
{\ensuremath{{\matrix}_{ \rowindex  \bcdot }^\T}}%
}

                                        % \inprod{ \mA, \mB } calculates matrix inner product
\noexpandarg
\DeclareDocumentCommand{\inprod}{ O{,} m }{%
\StrBefore{#2}{#1}[\leftmat]%
\StrBehind{#2}{#1}[\rightmat]%
{\ensuremath{  \left\langle \leftmat, \rightmat \right\rangle }}%
}






% -- text in math mode --
% \newcommand{\ppoly}[0]{{\rm poly}}
% \newcommand{\poly}[1]{\ensuremath{{\rm poly}\left(#1\right)}}
% \newcommand{\polylog}[1]{\ensuremath{{\rm polylog}\left(#1\right)}}
\renewcommand{\Pr}{\ensuremath{{\rm Pr}}}
\renewcommand{\th}{\ensuremath{^{\rm th}}}
\newcommand{\myexp}{\ensuremath{\mathbb{E}}}                    % expectation
\newcommand{\EE}{{\mathbb{E}}}

\newcommand{\const}{\mathop{\bf   const}\nolimits}
\newcommand{\diag}{\mathop{\bf    diag}\nolimits}
\newcommand{\grad}{\mathop{\bf    grad}\nolimits}
\newcommand{\Range}{\mathop{\bf   Range}\nolimits}
\newcommand{\rank}{\mathop{\bf    rank}\nolimits}
\newcommand{\supp}{\mathop{\bf    supp}\nolimits}
\newcommand{\trace}{\mathop{\bf   trace}\nolimits}
\newcommand{\tr}{\mathop{\bf      tr}\nolimits}                 % in one doc, you should stick to
                                                                % either trace or tr


% Bold first sentence of a floats caption using \bcaption{...}
% This solution only works when your sentence ends with a (single) period.
\makeatletter
\newcommand\@bff[1]{%
    \noexpandarg
    \IfSubStr{#1}{.}{%
      \StrBefore{#1}{.}[\firstcaption]%
      \StrBehind{#1}{.}[\secondcaption]%
      \textbf{\firstcaption.} \secondcaption}{%
      #1}%
}
\newcommand{\bcaption}[1]{\caption{\protect\@bff{#1}}}
\makeatother



% Smaller \circ How to "exchange" the \circ command with the
                                        % new one? Just replace the given code with
\let\latexcirc=\circ
\newcommand{\ccirc}{\mathbin{\mathchoice
  {\xcirc\scriptstyle}
  {\xcirc\scriptstyle}
  {\xcirc\scriptscriptstyle}
  {\xcirc\scriptscriptstyle}
}}
\newcommand{\xcirc}[1]{\vcenter{\hbox{$#1\latexcirc$}}}
\let\circ\ccirc


%
%
%
\newcommand\strong[1]{\textbf{#1}}
\newcommand\foreign[1]{\emph{#1}}

\newcommand\abbrev[1]{\textsc{\lowercase{#1}}\xspace}
\let\acronym\abbrev

\newcommand{\email}[1]{\href{mailto:#1}{\nolinkurl{#1}}}
\newcommand{\code}{\texttt}
\newcommand{\command}{\texttt}


% Define some colors
\newcommand\blue[1]{\color{blue}{{#1}}}
\newcommand\gray[1]{\color{lightgray}{{#1}}}

\let\grey\gray


\makeatletter
\newcommand\ensurecomma{\@ifnextchar,{}{\@latex@error{Don't forget the comma!}{}}}
\makeatother

%
% undefine some commands
%
\let\eg=\relax
\let\ie=\relax
\let\cf=\relax
\let\etc=\relax
\let\etal=\relax

\newcommand\eg{\foreign{e.g.}\ensurecomma}
\newcommand\ie{\foreign{i.e.}\ensurecomma}
\newcommand\cf{\foreign{cf.\xspace}}

                                        % The \@ ensures that no extra space is added after the
                                        % period as it would if the period ended a sentence.
\makeatletter
\newcommand\ensuresingleperiod{\@ifnextchar.{}{.\xspace}}
\makeatother

\newcommand\etc{\foreign{etc}\ensuresingleperiod}
\newcommand\etal{\foreign{et al}\ensuresingleperiod}


                                        % Elements of Typographic Style, recommends to use upright
                                        % parentheses in italic text xparse allows us to easily
                                        % define \emph to do what you asked for, and \emph* to do
                                        % the old version of \emph
\usepackage{expl3,xparse}               % These two are LaTeX3 packages
\ExplSyntaxOn
\cs_new_eq:Nc \emph_old:n { emph~ }     % Copying the old definition of `\emph`
\cs_new_eq:NN \emph_braces:n \textup    % Set up how braces should be typeset.
\cs_new:Npn \emph_new:n #1 {
  \tl_set:Nn \l_emph_tl {#1}
  \tl_replace_all:Nnn \l_emph_tl {(}{\emph_braces:n{(}}
  \tl_replace_all:Nnn \l_emph_tl {)}{\emph_braces:n{)}}
  \tl_replace_all:Nnn \l_emph_tl {[}{\emph_braces:n{[}}
  \tl_replace_all:Nnn \l_emph_tl {]}{\emph_braces:n{]}}
  \exp_args:NV \emph_old:n \l_emph_tl
}
\RenewDocumentCommand {\emph} {sm} {
  \IfBooleanTF {#1} {\emph_old:n {#2}} {\emph_new:n {#2}}
}
\ExplSyntaxOff
